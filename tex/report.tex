
\title{Rocketbear: A Forward Checking Constraint Solver for Binary and Unary
Constraints Written in Python}

\date{\today}

\author{120013913}

\documentclass{article}
\usepackage{amsmath}
\usepackage[pdftex]{graphicx}
% \usepackage{relsize}
\usepackage{float}
\usepackage{algorithm}
\usepackage{fancyvrb}
\usepackage[noend]{algorithmic}

\usepackage{url}

\usepackage{csvsimple}

\usepackage[margin=1.5in]{geometry}

\floatstyle{ruled} \newfloat{program}{thp}{lop} \floatname{program}{Structure}
\newcommand{\Normal}[3]{\mathcal{N}(#1, #2, #3)}
\newcommand{\Acronym}[1]{\ensuremath{{\small{\texttt{#1}}}}}
\newcommand{\True}{\Constant{true}}
\newcommand{\Symbol}[1]{\ensuremath{\mathcal{#1}}}
\newcommand{\Function}[1]{\ensuremath{{\small \textsc{#1}}}}
\newcommand{\Constant}[1]{\ensuremath{\small{\texttt{#1}}}}
\newcommand{\Var}[1]{\ensuremath{{\small{\textsl{#1}}}}}
\newcommand{\argmin}[1]{\underset{#1}{\operatorname{arg}\,\operatorname{min}}\;}
\newcommand{\argmax}[1]{\underset{#1}{\operatorname{arg}\,\operatorname{max}}\;}
\newcommand\numberthis{\addtocounter{equation}{1}\tag{\theequation}}

\begin{document}

\maketitle

\section{Introduction}

In this practical we are asked to develop a forward checking constraint solver
that uses d-way branching in a language of our choosing. We are also asked to
decide on a representation for the constraint and a mechanism for undoing
pruning upon backtracking. We were also tasked with implementing various
heuristics for variable orderings and to provide empirical evidence on their
merit. In order to measure the performance of the solver and the heuristics, we
are asked to choose a program class to perform benchmark tests on.

I have used the Python programming language to implement this constraint solver
dubbed Rocketbear. I have provided support for unary constraints on variables
and binary constraints between variables. Three dynamic and two static
heuristics have been implemented. The dynamic heuristics include highest degree
first, minimum domain cardinality first, and minimum ratio of domain
cardinality to degree.  Static analogues of these heuristics are provided for
the former two.  I have chosen to use graph colouring as the problem instance
to benchmark the solver and have used a variety of sample graphs found online
with the implemented heuristics to provide evidence of the solver's
performance. This report is structured by first describing the representation
of variables by Rocketbear, then the representation of constraints, the
heuristics, transforming graph colouring into a Rocketbear constraint graph,
the results, and a discussion.

\section{Variables}

Rocketbear represents variables using one's full domain, pruned domain, name,
assignment, and unary constraints. The full domain is all the possible values
that the variable can be assigned. The pruned domain is dynamic and changes as
forward checking progresses. The name of the variable can be anonymous and
automatically assigned by the library or can be explicitly stated. The
assignment is also a dynamic feature that is used by the forward checking
solver to update the variables value. The unary constraints on a variable are
added once the variable has been instantiated.

\section{Constraints}

Both binary and unary constraints are supported by Rocketbear. Unary
constraints are added directly to each variable. A unary constraint can be seen
as a function that takes in a variable assignment and returns true if the
constraint is satisfied. A variable may have more than one unary constraint and
these are added into a list of unary constraint maintained by each variable.
Binary constraints are represented by functions which take in two parameters,
the left and right assignments of the variables on an arc, and returns true if
the constraint is satisfied. An arc between variables $i$ and $j$ is added to
the constraint graph if there is at least one constraint between $i$ and $j$.
The binary constraint between $i$ and $j$ is stored in the arc object between
$i$ and $j$.

Besides simple binary and unary constraints, the macro-constraint,
$\Var{AllDifferent}$ was added. The $\Var{AllDifferent}$ constraint synthesizer
took an arbitrary number of variables in as parameters, and added a not equals
binary constraint from each variable to every other variable in the constraint
graph. Even though the $\Var{AllDifferent}$ is not used in the graph colouring
transformation described in Sec.~\ref{sec:colouring}, an example is shown in
\verb|examples/solver_example.py|.

Also, minimizing capabilities are provided by Rocketbear by specifying a number
$k$ such that all variables must be less than $k$. This is done by adding unary
constraints on all variables indicating that their value must be less than a
given value $k$. It is also possible to use unary constraints to specify
minimization properties for subsets of variables.

\section{Heuristics}

Five heuristics have been implemented on the variable ordering.  Three are
dynamic and two are static. A static heuristic provides a single unchanging
ordering for the variables that are expanded in forward checking. A dynamic
ordering is dependent on features such as the size of the pruned domains, or
the number of valid edges in the constraint graph left for a variable. This
dynamic orderings change as the search progresses. The five heuristics are
described below.

\subsection{Static Most Arcs First}

This heuristic will order the variables based on the number of arcs in the
constraint graph a certain variable has. Specifically, this heuristic expands
variables with the most number arcs (i.e. the most number of binary
constraints) first in order to reduce the size of the search space.

\subsection{Static Smallest Domain First}

This heuristic will order the variables based on the cardinality of their
initial domains.  Specifically the variable that has the largest initial domain
will be expanded first when expanding the search tree.

\subsection{Dynamic Most Arcs First}

This heuristic is a variant of its static counter part, except with on
difference. Instead of counting the total number arcs leading to a variable and
letting that determine its ordering, only the number of arcs that have not yet
been assigned count towards the ordering. Therefore if a node has five arcs
associated with it but three have already been assigned, the evaluation of the
heuristic will be two since this heuristic does not take into account
neighbouring variables that have already been assigned.

\subsection{Dynamic Smallest Domain First}

This heuristic is similar to the static smallest domain first, except instead
of counting the full domain of a variable and letting that determine its
heuristic evaluation, the pruned domain is used. This means that a variable
that originally had a domain cardinality of ten but has been pruned and revised
down to three, the heuristic evaluation of the variable will be three.

\subsection{Dynamic Smallest Domain Over Degree First}

This heuristic is a combination of the static most arcs first and the dynamic
smallest domain first heuristics. This heuristic evaluates the cardinality of
the pruned domain and divides it by the amount of arcs in total a variable has.
Since static most arcs first heuristic gives higher heuristic values for
variables with more arcs and the dynamic smallest domain first gives higher
heuristic evaluations for smaller cardinality domains, by dividing these two
heuristics, specifically the cardinality of the pruned domain by the number of
arcs a variable has, this heuristic is a good combination of two other
indicators for variable ordering.

\section{Graph Colouring}

\label{sec:colouring}

The instance problem I have chosen to experiment on is graph colouring. In
order to perform graph colouring using Rocketbear, the graph to be coloured is
converted in to a constraint graph. This is done by creating an empty
constraint graph, iterating over the edges for the graph to be coloured, and
adding a constraint indicating that the two nodes on the edge of the graph
cannot be equal to the constraint graph. The domains of the variables in the
newly generated constraint graph are all equal in cardinality and value and are
comprised of the integers from 0 to $k$ where $k$ is the maximum amount of
colours to be used in the graph colouring and is given by the user when the
colouring method is called. If the graph cannot be coloured using $k$ colours,
the method returns a null pointer.

\section{Results}

\label{sec:results}

For the experiments, three different graphs representing the $n$-queens problem
are used. Each of the five heuristics for variable ordering are tested on each
graph. The number of nodes expanded, the number of times backtracking is
needed, the number of assignments, and the amount of time it took to find a
solution are gathered for each heuristic on each graph. Each one of the
experiment instances ran as a separate process on the school server,
\verb|lyrane.cs.st-andrews.ac.uk|. This data from the different graphs using
varying heuristics is shown in the tables below. The acronyms are used to
distinguish between heuristics. If a heuristic does not have a value in the
table, it exceeded the time-out of one hour. For each of the tables, the
corresponding graph is located in the \verb|graphs/| directory.

From the tables it is apparent that the Dynamic Smallest Domain First heuristic
expands the least amount of nodes, uses the least amount of assignments, and
backtracks the least out of all of the graph instances that were tested.
Likewise, the for all but one graph instance, it produced a solution in the
last amount of time. For the Queen5\_5 graph, the Static Most Arcs First
heuristic produced a solution in the smallest amount of time, however for the
other two graph instances, it either did not produce a solution within an hour
or took the longest to solve. Likewise for the first graph, the Dynamic Most
Arcs First heuristic took by far the longest amount of time to produce solution
and it expanded the most nodes, had the most assignments, and the most
backtracks. For the second graph instance, the Static Smallest Arcs First had
the most assignments, nodes, and backtracks and took the longest to compute.
For the last graph instance, none of the heuristics using solely the arcs were
able to produce a solution in the provided time frame.

From these tables, it is apparent that the Dynamic Smallest Domain First
heuristic produces solutions in the least amount of time, by having the least
amount of assignments, nodes expanded, and backtrackings. The Static Smallest
Domain First and Dynamic Domain Over Degree heuristics produced the same
assignments, nodes, and number of times backtracking was needed for each of the
graph instances. They also took roughly the same amount of time for each
instance.

\begin{figure}[ht]
    \centering
    \csvautotabular{../data/queen5_5.col.dat}
    \caption{Table showing data for the Queen5\_5 graph}
    \label{fig:5}
\end{figure}

\begin{figure}[h!]
    \centering
    \csvautotabular{../data/queen6_6.col.dat}
    \caption{Table showing data for the Queen6\_6 graph}
    \label{fig:6}
\end{figure}

\begin{figure}[h!]
    \centering
    \csvautotabular{../data/queen7_7.col.dat}
    \caption{Table showing data for the Queen7\_7 graph}
    \label{fig:7}
\end{figure}

\section{Discussion}

From the results shown in Sec.~\ref{sec:results}, it is shown that for
different instances of graph colouring, Dynamic Smallest Domain First, produces
the solutions the fastest with the least number of assignments, the smallest
number of nodes expanded during search, and the least number of times
backtracking was needed. The heuristics involving solely the number of arcs
produced the worst results by having the most assignments, the most number of
nodes expanded during search, and the most number of times backtracking was
needed.

In this practical, all of the basic requirements are met including binary
constraints, minimization, and detailed experimentation using a popular problem
class. Static ordering variables is allowed by the constraint solver as well as
smallest domain first orderings. As an extension, unary constraints are added
to the variables and dynamic ordering heuristics are provided.

Rocketbear has shown to be effective at providing solutions to graph colouring
using a constraint solver, and has shown that certain heuristics produce
solutions faster and less costly than other heuristics.

\end{document}
